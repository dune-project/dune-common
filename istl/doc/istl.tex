\documentclass[11pt]{article}
\usepackage{multicol}
\usepackage{ifthen}
%\usepackage{multitoc}
%\usepackage{german}
%\usepackage{bibgerm}
\usepackage{amsmath}
\usepackage{amsfonts}
\usepackage{color}
\usepackage{hyperref}
\usepackage[dvips]{epsfig}
\usepackage[dvips]{graphicx}
\usepackage[a4paper,body={154mm,240mm,nohead}]{geometry}
\usepackage[ansinew]{inputenc}
\usepackage{listings}
\lstset{language=C++, indent=20pt, basicstyle=\small\ttfamily,
  stringstyle=\ttfamily, commentstyle=\it, extendedchars=true}

\newif\ifpdf
\ifx\pdfoutput\undefined
\pdffalse % we are not running PDFLaTeX
\else
\pdfoutput=1 % we are running PDFLaTeX
\pdftrue
\fi

\ifpdf
\usepackage[pdftex]{graphicx}
\else
\usepackage{graphicx}
\fi

\ifpdf
\DeclareGraphicsExtensions{.pdf, .jpg, .tif}
\else
\DeclareGraphicsExtensions{.eps, .jpg}
\fi

\newcommand{\C}{\mathbb{C}}
\newcommand{\R}{\mathbb{R}}
\newcommand{\N}{\mathbb{N}}
\newcommand{\Z}{\mathbb{Z}}
\newcommand{\Q}{\mathbb{Q}}
\newcommand{\K}{\mathbb{K}}


\title{Iterative Solver Template Library\thanks{Part of the
    Distributed and Unified Numerics Environment (DUNE) which is
    available from the site
    \texttt{http://www.dune.uni-hd.de/}}}

\author{%
Peter Bastian\\
Interdisziplin�res Zentrum f�r Wissenschaftliches Rechnen,\\
Universit�t Heidelberg, Im Neuenheimer Feld 368, D-69120 Heidelberg, \\
email: \texttt{Peter.Bastian@iwr.uni-heidelberg.de}}

\date{\today}

\begin{document}

\maketitle

\begin{abstract}
This document describes the rationale behind and use of the Iterative
Solver Template Library (ISTL) which provides a set of C++
templates to represent vectors, (sparse) matrices and some generic
algorithms based on these. The most prominent features of the
matrix/vector classes is that they support a recursive block structure
in a bottom up way. The classes can be used, e.~g., to efficiently implement
block preconditioners for $hp$-finite elements.
\end{abstract}

\begin{multicols}{2}
{\small\tableofcontents}
\end{multicols}

\section{Introduction}

The numerical solution of partial differential equations frequently requires the
solution of large and sparse linear systems. Naturally,
there are many libraries available on the internet for doing sparse matrix/vector
computations. A comprehensive overview is given in \cite{LALinks}. 

The
widely availably Basic Linear Algebra Subprograms (BLAS) standard has
been extended to cover als sparse matrices \cite{BLASTForum}. BLAS
divides the available functions into level 1 (vector operations),
level 2 (vector/matrix operations) and level 3 (matrix/matrix
operations). BLAS for sparse matrices contains only level 1 and 2
functionality and is quite different to the standard for dense
matrices. The standard uses procedural programming style and offers
only a FORTRAN and C interface. As a consequence, the interface is
``coarse grained'', meaning that ``small'' functions such as access to
individual matrix elements is relatively slow. 

Generic programming techniqes in C++ offer the possibility to combine
flexibility and reuse (``efficiency of the programmer'') with fast
execution (``efficieny of the program'') as has been demonstrated with
the Standard Template Library (STL), \cite{Stroustrup} or the Blitz++
library for multidimensional arrays \cite{Blitz}. A variety of
template programming techniques such as traits, template metaprograms,
expression templates or the Barton-Nackman trick are used in the
implementations, see \cite{BN,Veldhui99} for an introduction.
Application of these ideas to matrix/vector operations is available
with the Matrix Template Library (MTL), \cite{MTL}. The Iterative
Template Library (ITL), \cite{ITL}, implements iterative solvers for
linear systems (mostly Krylov subspace methods) in a generic way based
on MTL. The Distributed and Unified Numerics Environment (DUNE),
\cite{Dune,DuneWeb}, applies the STL ideas to finite element
computations. 



\section{Vectors}

You can define a variable block size vector via
\begin{lstlisting}{}
const int N=1;
typedef Dune::FieldVector<double,N> RN;

typedef Dune::VariableBlockVector<RN> Vector;

Vector x(20); // make a vector with 20 blocks

for (Vector::CreateIterator i=x.createbegin(); i!=x.createend(); ++i)
  i.setblocksize((i.index()%10)+1);
\end{lstlisting}
and then get the result
\begin{lstlisting}{}
... bla
\end{lstlisting}

\subsection{Making a vector from a field}

\section{Matrices}

\section{Algorithms}

% bibtex bibliography
\bibliographystyle{plain}
\bibliography{istl.bib}

% some links
% http://www.netlib.org/blas/blast-forum/
% http://www.osl.iu.edu/research/mtl/
% http://www.osl.iu.edu/research/itl/ (based on MTL, mostly Krylov)
% http://netlib.org/utk/people/JackDongarra/la-sw.html
% the last one: FREELY AVAILABLE SOFTWARE FOR
% LINEAR ALGEBRA ON THE WEB (May 2004) by Jack Dongarra
% http://www.oonumerics.org/blitz/

\end{document}

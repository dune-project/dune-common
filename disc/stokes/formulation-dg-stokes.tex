\documentclass[11pt,a4paper]{article}
%%\defaulthyphenchar=-1 % suppress hyphenation of words.
\pretolerance=10000 % to intelligent way of hyphenation
\hyphenpenalty=10000 % to intelligent way of hyphenation
\topmargin -1.5cm        % read Lamport p.163
 \oddsidemargin -0.04cm   % read Lamport p.163
 \evensidemargin -0.04cm  % same as oddsidemargin but for left-hand pages
 \textwidth 16.59cm
 \textheight 21.94cm 
%\pagestyle{empty}       % Uncomment if don't want page numbers
 \parskip 5.2pt           % sets spacing between paragraphs
 %\renewcommand{\baselinestretch}{1.5}  % Uncomment for 1.5 spacing between lines
 \parindent 0pt           % sets leading space for paragraphs
\usepackage{palatino,url}
\usepackage{amsmath}
\usepackage{amstext}
\usepackage{array}
\usepackage{graphics}
\usepackage{listings}
\begin{document}

\title {Discontinuous Galerkin Discretization of Stokes Equation} 
\author {Sreejith P. Kuttanikkad}
\maketitle

\section{The Problem}
Stationary  Stokes problem for an incompressible viscous fluid can be used 
for representing slow viscous dominated flow in the tiny pores in a porous media, 
which is characteristic at the pore-scale. The equation is given by,\\
Find velocity $\bf{u}$ and pressure $p$ such that 
\begin{eqnarray}
-\mu \Delta \bf{u}+ \nabla p &=& f \qquad\qquad\text{in} \qquad \Omega \\
\nabla \cdot \bf{u} &=& 0 \qquad\qquad\text{in}\qquad \Omega\\
\bf{u} &=& \bf{g} \qquad\qquad\text{on}\qquad\partial \Omega
\end{eqnarray}
The (1) is the momentum equation, 
(2) is the mass conservation or incompressibility condition and 
(3)  Dirichlet boundary condition. viscosity of fluid $\mu > 0 $. \\
(note: what about pressure boundary condition?? normalization of pressure assumed over the region? hows this incorporated while code implementing?) still not yet clear.\\

(some points are mentioned in page 8,9 of paper3; page 2 of paper3, page 6 of paper1)

Consider 2D Stokes Equations.\\

let $u_{x}$ and $u_{y}$ be the horizontal and vertical velocity components
and $p$ be the pressure.then\\
\begin{equation*}
\begin{aligned}
- \mu~\Delta u_{x} + {\partial p \over \partial x} &= f1\\
- \mu~\Delta u_{y} + {\partial p \over \partial y} &= f2\\
{\partial u_{x} \over \partial x}+{\partial u_{y} \over \partial y}&=0
\end{aligned}
\end {equation*}
\newpage

{Stokes equations are given by:}
\begin{eqnarray*} 
 -\mu~{\Delta{u}}+{\nabla{p}} &=& f \qquad\qquad\text {in} \qquad \Omega \\
\nabla{\cdot{u}} &=& 0 \qquad\qquad\text{in} \qquad \Omega \\
u &=& g \qquad\qquad\text{on}\qquad \partial \Omega
     \end{eqnarray*}

{Standard weak formulation of the Stokes equation:}
{
Find $(u,p) \in V~\times~Q $ 
$$
\left\{
  \begin{array}{lcl}
   \forall v \in V,~ \mu~(A(u,v)) + B(v,p) &=& \int_{\Omega} f \cdot v~ dx  \\
   \forall q \in Q,~ B(u,q)  &=& 0
  \end{array}
 \right.
$$
}


{DG approximation of the Stokes problem:}
{{
Find $(u,p) \in V~\times~Q $ 
$$
\left\{
  \begin{array}{lcl}
   \forall v \in V,~ \mu~(A(u,v)) + B(v,p) &=& \int_{\Omega} f \cdot v~ dx  \\
   \forall q \in Q,~ B(u,q)  &=& 0
  \end{array}
 \right.
$$
where the bilinear forms are,
\begin{eqnarray*}
A(u,v) & = & \sum\limits_E\int_\Omega \nabla u \cdot \nabla v -
\sum\limits_{\gamma_{\text{ef}} \in \Gamma_{\text{int}}} \int_{\gamma_{\text{ef}}}\langle
\nabla u \cdot n \rangle [v] + \epsilon \sum\limits_{\gamma_{\text{ef}} \in \Gamma_{\text{int}}} \int_{\gamma_{\text{ef}}} \langle \nabla v
\cdot n \rangle [u]\\
B(v,p) & = & - \sum \int_\Omega p \nabla \cdot v + \sum\limits_{\gamma_{\text{ef}} \in \Gamma_{\text{int}}} \int_{\gamma_{\text{ef}}}
\langle p \rangle [v \cdot n]\\
B(u,q) & = & - \sum \int_\Omega q \nabla \cdot u + \sum\limits_{\gamma_{\text{ef}} \in \Gamma_{\text{int}}} \int_{\gamma_{\text{ef}}}
\langle q \rangle [u \cdot n]\\ 
\end{eqnarray*}
}
}



\newpage
the exact problem that i would like to test on:\\
\begin{equation*}
\begin{aligned}
- \mu~\Delta u + {\partial p \over \partial x} &= f1\\
- \mu~\Delta v + {\partial p \over \partial y} &= f2\\
{\partial u \over \partial x}+{\partial v \over \partial y}&=0
\end{aligned}
\end{equation*}

we could test on a simple rectangular grid. solution will be a parabolic velocity profile and a linear ramp of pressure.
exact values are:\\
$u=x^2$, $v=-2xy$ , $p=x$ , $f1=-1$ , $f2=0$ \\
u,v are x \& y comp\\ 
in the implementation, i use a second order monomial basis for velocity and a linear pressure basis.\\

the DG formulation in detail is given in the next section.

\clearpage
{{The bilinear form $A(u,v)$ can be chosen differently and accordingly  results in different DG schemes such as SIPG,NIPG,OBB etc.}} \\
{Symmetric Interior Penalty Galerkin(SIPG), \quad$\epsilon = -1$ \& $\sigma > 0$}\\
u - velocity vector\\
p- pressure\\
v,q - test velocity and test pressure\\


DG schemes for Stokes problem\\

$\sigma > 0$ and $\epsilon = +1$ (NIPG)\\
$\sigma > 0$ and $\epsilon = -1$ (SIPG)\\
$\sigma = 0$ and $\epsilon = +1$ (OBB)\\


Find $(u,p) \in V~\times~Q $ 
\vspace{-0.45cm}
$$\left\{
  \begin{array}{lcl}
   \forall v \in V,~ \mu~(A(u,v)+{J(u,v)}) + B(v,p) &=& F(v)  \\
   \forall q \in Q,~ B(u,q)  &=& G(q)
  \end{array}
 \right.$$
\vspace{-0.75cm}
\begin{eqnarray*}
A(u,v) & = & \sum\limits_E\int_\Omega \nabla u \cdot \nabla v -
\sum\limits_{\gamma_{\text{ef}} \in \Gamma_{\text{int}}} \int_{\gamma_{\text{ef}}}\langle
\nabla u \cdot n \rangle [v] + {\epsilon} \sum\limits_{\gamma_{\text{ef}} \in \Gamma_{\text{int}}} \int_{\gamma_{\text{ef}}} \langle \nabla v
\cdot n \rangle [u]\\
 && - \sum\limits_{\gamma_{\text{e}} \in \Gamma_{D}}\int_{\gamma_{\text{e}}}  (\nabla u \cdot  n)  v + {\epsilon}
\sum\limits_{\gamma_{\text{e}} \in \Gamma_{D}}\int_{\gamma_{\text{e}}} (\nabla v \cdot n) u\\
J(u,v) & = & {\sum\limits_{\gamma_{\text{ef}} \in \Gamma_{\text{int}}} \frac{\sigma}{|{e}|}\int_{\gamma_{\text{ef}}} [u] \cdot [v]} + 
{\sum\limits_{\gamma_{\text{e}} \in \Gamma_{\text{D}}} \frac{\sigma}{|{e}|}\int_{\gamma_{\text{e}}} u \cdot v}\\
B(v,p) & = & - \sum \int_\Omega p \nabla \cdot v + \sum\limits_{\gamma_{\text{ef}} \in \Gamma_{\text{int}}} \int_{\gamma_{\text{ef}}}
\langle p \rangle [v \cdot n] + \sum\limits_{\gamma_{\text{e}} \in \Gamma_{\text{D}}} \int_{\gamma_{\text{e}}} p ~ v \cdot n\\
B(u,q) & = & - \sum \int_\Omega q \nabla \cdot u + \sum\limits_{\gamma_{\text{ef}} \in \Gamma_{\text{int}}} \int_{\gamma_{\text{ef}}}
\langle q \rangle [u \cdot n]+ \sum\limits_{\gamma_{\text{e}} \in \Gamma_{\text{D}}} \int_{\gamma_{\text{e}}} q ~ u \cdot n\\
F(v) &=& \sum\limits_E \int_{\Omega} f \cdot v + \mu~{\epsilon}
\sum\limits_{\gamma_{\text{e}} \in \Gamma_{D}}\int_{\gamma_{\text{e}}} (\nabla v \cdot n) g+
\mu~{\sum\limits_{\gamma_{\text{e}} \in \Gamma_{\text{D}}} \frac{\sigma}{|{e}|}\int_{\gamma_{\text{e}}} g \cdot v}\\
G(q) &=& \sum\limits_{\gamma_{\text{e}} \in \Gamma_{\text{D}}} \int_{\gamma_{\text{e}}} q ~ g \cdot n
\end{eqnarray*}



\newpage
$v$ and $q$ are test functions\\
let the expansion basis be

\begin{eqnarray*}
u=\sum c^e_i \varphi^e_i\\ 
v=\sum c^e_j \varphi^e_j\\
p=\sum c^e_i \psi^e_i\\
q=\sum c^e_j \psi^e_j
\end{eqnarray*}

($e$ and $f$ are adjescent elements). The direction of unit normal vetor $n$ is determined by the numbering of elements. 
the direction of this vector is taken to be from element $e$ to $f$ if $e>f$ \\

Term 1
\begin{equation*}
\mu \sum\limits_{e} \int\limits_{\Omega_e} \nabla u \cdot \nabla v~ds
\end{equation*}
Term 2
\begin{equation*}
- \mu \sum\limits_{\gamma_{\text{ef}} \in \Gamma_{\text{int}}} \int_{\gamma_{\text{ef}}}\langle
\nabla u \cdot n \rangle [v]
\end{equation*}
Term 3
\begin{equation*}
- \mu \sum\limits_{\gamma_{\text{e}} \in \Gamma_{D}}\int_{\gamma_{\text{e}}}  (\nabla u \cdot  n)  v
\end{equation*}
Term 4
\begin{equation*}
\mu {\epsilon} \sum\limits_{\gamma_{\text{ef}} \in \Gamma_{\text{int}}} \int_{\gamma_{\text{ef}}} \langle \nabla v
\cdot n \rangle [u]
\end{equation*}
Term 5
\begin{equation*}
\mu {\epsilon} \sum\limits_{\gamma_{\text{e}} \in \Gamma_{D}}\int_{\gamma_{\text{e}}} (\nabla v \cdot n) u
\end{equation*}
Term 6
\begin{equation*}
\mu {\sum\limits_{\gamma_{\text{ef}} \in \Gamma_{\text{int}}} \frac{\sigma}{|{e}|}\int_{\gamma_{\text{ef}}} [u] \cdot [v]}
\end{equation*}
Term 7
\begin{equation*}
\mu {\sum\limits_{\gamma_{\text{e}} \in \Gamma_{\text{D}}} \frac{\sigma}{|{e}|}\int_{\gamma_{\text{e}}} u \cdot v}
\end{equation*}
Term 8
\begin{equation*}
- \sum \int_\Omega p \nabla \cdot u 
\end{equation*}
\newpage
Term 9
\begin{equation*}
\sum\limits_{\gamma_{\text{ef}} \in \Gamma_{\text{int}}} \int_{\gamma_{\text{ef}}}
\langle p \rangle [v \cdot n]
\end{equation*}
Term 10
\begin{equation*}
 \sum\limits_{\gamma_{\text{e}} \in \Gamma_{\text{D}}} \int_{\gamma_{\text{e}}} p ~ v \cdot n
\end{equation*}
Term 11
\begin{equation*}
- \sum \int_\Omega q \nabla \cdot u 
 \end{equation*}
Term 12
\begin{equation*}
\sum\limits_{\gamma_{\text{ef}} \in \Gamma_{\text{int}}} \int_{\gamma_{\text{ef}}}
\langle q \rangle [u \cdot n]
\end{equation*}
Term 13
\begin{equation*}
\sum\limits_{\gamma_{\text{e}} \in \Gamma_{\text{D}}} \int_{\gamma_{\text{e}}} q ~ u \cdot n
\end{equation*}
Term 14
\begin{equation*}
\sum\limits_E \int_{\Omega} f \cdot v
\end{equation*}
Term 15
\begin{equation*}
\mu~{\epsilon}\sum\limits_{\gamma_{\text{e}} \in \Gamma_{D}}\int_{\gamma_{\text{e}}} (\nabla v \cdot n) g
\end{equation*}
Term 16
\begin{equation*}
\mu~{\sum\limits_{\gamma_{\text{e}} \in \Gamma_{\text{D}}} \frac{\sigma}{|{e}|}\int_{\gamma_{\text{e}}} g \cdot v}
\end{equation*}
Term 17
\begin{equation*}
\sum\limits_{\gamma_{\text{e}} \in \Gamma_{\text{D}}} \int_{\gamma_{\text{e}}} q ~ g \cdot n
\end{equation*}


\newpage
matrix assembling\\

 $u$ , $v$ , $p$ be the unknowns:\\

assembling terms:\\
u - velocity row: $u$$u$ coupling and $u$$p$ coupling\\
v - velocity row: $v$$v$ coupling and $v$$p$ coupling\\
p - pressure row: $p$$u$ coupling and $p$$v$ coupling\\

\newpage
Matrix contribution: Term 1\\
\begin{equation*}
\begin{aligned}
\mu \sum\limits_{e} \int\limits_{\Omega_e} \nabla u : \nabla v~ds\\
\mu \sum\limits_{e} \int\limits_{\Omega_e} \nabla \varphi^e_i \cdot \nabla \varphi^e_j\\ 
\end{aligned}
\end{equation*}

\begin{lstlisting}{}
 
\end{lstlisting}
\newpage
Matrix contribution: Term 17\\
\begin{equation*}
\sum\limits_{\gamma_{\text{e}} \in \Gamma_{\text{D}}} \int_{\gamma_{\text{e}}} q ~ g \cdot n
\end{equation*}
\begin{lstlisting}{}
  
\end{lstlisting}
\newpage
Matrix contribution: Term 11\\
\begin{equation*}
- \sum \int_\Omega q \nabla \cdot u 
\end{equation*}
\begin{lstlisting}{}

\end{lstlisting}
\newpage
Matrix contribution: Term 13\\
\begin{equation*}
\sum\limits_{\gamma_{\text{e}} \in \Gamma_{\text{D}}} \int_{\gamma_{\text{e}}} q ~ u \cdot n
\end{equation*}
\begin{lstlisting}{}

\end{lstlisting}
%-------------------------------%

\end{document}


